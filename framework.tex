\section{Framework}
We design the architecture of the experimental framework that we will use during the project for conducting perceptual experiments. We identify the following entities that can interact together in numerous way. A \textbf{performance} occurs when two or more \textbf{subjects} interact together through a \textbf{medium}. Subjects can be musicians during a rehearse, as well as teacher and student for a front lesson. Subjects interact from an \textbf{environment}, that is, the physical location where they perform, which has their own timbral (acoustic of the room) and spatial properties (location of the subjects in the room). If the environments are different, the subjects will interact through a \textit{networked medium}, hence having a NMP; otherwise they interact through a \textit{physical medium}. The comparison between a networked and physical medium is crucial to understand how to design the interaction so that the \textit{virtual environment} perceived from the other end of the medium matches the expectations of a real environment. In order to analyze the performance, it is crucial to acquire multimodal signals by means of  \textbf{acquisition devices}. The factors and aspects that will be possible to analyze from the performance depend on the properties of the devices, e.g., whether they are \textit{video} or \textit{audio devices}, or where they are placed. In the following sections we detail the aforementioned entities.

%In this paper we use the description of the framework as an abstraction of the perceptual experiments we have conducted and of the aspects of NMPs we aim to investigate. Moreover, we intend to further develop this formalization into an ontology, which will be integrated in the NMP tools, in order to collect and analyze a number of semantically-annotated rehearsal or lessons. 

\subsection{Performance}
\subsection{Subjects}
\subsection{Environment}
\subsection{Medium}
\subsection{Acquisition devices}

