
\section{Introduction}\label{sec:introduction}
With the rapid evolution of technology and the increase speed of network connections, numerous applications have become possible.  

Computer-aided musical collaborations between geogra\-phically-displaced musicians have been subject of extensive investigation from a variety of perspectives, since the late '90s. Early categorizations of computer systems for musical interaction have been proposed in~\cite{barbosa2003displaced}, based on the temporal (synchronous vs. asynchronous) and spatial (co-located vs. remote) dimensions of the %musical
performance. 
%Typically, the research and development focused 
The research and development has focused 
on the technical and perceptual issues (i.e., network delay and audio quality) affecting the on-line, simultaneous performance between musicians located and playing in remote rooms~\cite{rottondi2016overview}. 

A substantial body of work has been produced in that area that has been crystallized under the acronym of NMP, namely Networked Music Performance. Gabrielli and colleague provide a valuable picture of the state of the art of NMP research and projects~\cite[Chapter~2, 3]{gabrielli2016networked}.

From a different angle, a renovated interest in remote collaborative environments has been growing in the area of audio-video (AV) streaming and conferencing systems for educational purposes~\cite{alpiste2013telepresence}. NMP technologies and tools are increasingly being available on the market, and proposed as viable standards in blended and distance learning~\cite{IorwerthNMP2015}. 

The EU funded project InterMUSIC\footnote{\url{http://intermusicproject.eu/}. The Consortium is composed by the Conservatory of Music ``G. Verdi'' of Milano (Coordinator), the Image and Sound Processing Group of the Polytechnic University of Milan, the RDAM Royal Danish Academy of Music of Copenhagen, the LMTA Lithuanian Academy of Music and Theatre of Vilnius, the AEC Association Europ{\'e}enne des Conservatoires, Acad{\'e}mies de Musique et Musikhochschulen.} (Interactive Environment for Music Learning and Practicing, 2017 - 2020) aims to bridge the approaches established in NMP research with the opportunities of distance learning and education. 

A project-based and practice-led approach to research %~\cite{frankel2010complex} 
is aimed at distilling, by the end of the project, an effective and operational environment, and especially a systematized knowledge in the form of best practices and guidelines for the implementation of remote environments for music interaction and education. This involves to author three online pilot courses in music theory and composition, chamber music practice, and vocal training, by means of the implementation of Massive Open Online Courses (MOOCs). With MOOCs, the interaction between teacher and students is usually one-to-many and asynchronous. On the other side, distance learning of the music practice can be delivered through one-to-one synchronous interaction between the teacher and the student. 
\begin{figure*}[!t]
	\centerline{
		\includegraphics[clip,trim={4.6cm 8cm 5cm 8cm},width=\textwidth]{figures/instrumentalists.pdf}}
	%    \vspace{-.5em}
	\caption{Instrumentalist positioning in room 1, with frontal view of the co-performer displaced in room 2.} 
	\label{fig:instrumentalists}
	%    \vspace{-1em}
\end{figure*}

One of the most popular tool for NMP is designed to provide low-latency interaction by means of dedicated connections and high-performing hardware \cite{drioli2013networked}. In the common scenario, however, students do not have access to such networks and need to connect with the teacher by means of general purpose connections and hardware, which introduce processing and transmission latency.  

With regard to NMP, the goal of the project is to investigate the user experience in order to optimize and improve the tools currently available. In this paper, we introduce a study, still in progress, aimed at understanding how temporal factors (i.e., network latency) affect the sense of presence, and the quality of the performance of chamber music duos involved in remote collaboration, i.e music making. 

We ask duos to perform a short exercise, under diverse conditions of network delay. 
The exercise is specifically conceived around musical structures which are functional to pinpointing a set of problems relative to time management, communication mechanisms and mutual understanding between remote performers. A qualitative assessment through questionnaires on the sense of presence and the perceived quality of the performance~\cite{schubert2001experience} is combined with quality metrics of the objective performance~\cite{rottondi2015feature}.
A follow-up study will be devoted instead to the investigation of spatial representations, auditory and visual.

The premise is that effective music-making and communication rely on the availability of auditory and visual cues (i.e., sonic gestures)~\cite{godoy2010musical}, which are inevitably constrained in NMP, and in telepresence environments in general. 
We turn the traditional engineering approach to NMP research upside down, and seek for design strategies to compensate, and facilitate a plausible music experience in the mediated environment. 

In this respect, we are investigating how simulated network delay, and diverse modes of audio-visual spatial representation, separately, affect the subjective experience of being present and together in the shared reality environment. 
Sensory breadth and depth, degree of control and anticipation of events, together with the overall interactivity of the environment, represent crucial elements in both presence and performance, being the first a prerequisite for the second~\cite{nash2000review}.

%The paper is organized as follows: in Section~\ref{sec:relatedwork} we provide the background context on top of which we are building our study; in Section~\ref{sec:presence} we elaborate on the concept of presence in the embodied cognition framework; Section~\ref{sec:experiment} introduces the pilot experiment, and the methodology; we discuss the preliminary observations collected, in Section~\ref{sec:discussion}.
