
\section{Introduction}\label{sec:introduction}

The rapid evolution of technology and the increasing speed of network connections have allowed to dramatically reduce the distances and see new scenarios of interaction among geogra\-phically-displaced people. Since the late '90s, a community composed of musicians, technicians and  scientists has been investigating how to use technology to enable music performances leading to the definition of Networked Music Performances (NMPs), which \textit{occurs when a group of musicians, located at different physical locations, interact over a network to perform as they would if located in the same room} \cite{Lazzaro2001}.

NMPs have immediate application in the performative scenario, enabling musicians to rehearse from remote distances and perform together in geographically-distributed concerts [CIT.]. As part of the broader category of audio-video streaming and conferencing systems, NMPs have been interesting applications of educational purpose such as blended and distance learning \cite{IorwerthNMP2015}.    


% InterMUSIC: what it is, what it aims, how it is involved with NMP.
The EU funded project InterMUSIC\footnote{http://intermusicproject.eu/} (Interactive Environment for Music Learning and Practicing, 2017 - 2020) aims to develop and improve tools for distance learning of music and to collect such tools in integrated remote environments for music interaction and education. The educational scenarios that have been chosen are three online pilot courses in music theory and composition, chamber music practice, and vocal training, to be implemented by means of a platform for Massive Open Online Courses (MOOC), with possible interactive exercises, and a software for NMP, in the light of the specific requirements for master-student lessons as well as rehearsal. While the former involves mainly asynchronous activities and will be based as a one-to-many communication, the latter has stricter requirements in terms of one-to-one (or few-to-few) synchronous communication. 

%Music Networked Performance is an interesting outcome of technology, and has receive attention under different factors.
The literature about NMPs has investigated numerous factors which may affect several aspects of the performances. Affecting factors include the unavoidable network latencies [CIT], timbral properties of the employed instruments and rhythmic properties of the performed pieces [CIT]; while affected aspects range from objective quality of the performance [CIT], to perceptual metrics of musicians' comfortability [CIT nostre]. This investigation has led a highly popular tool for NMP, which allows low-latency interaction by using high-end hardware connected through ad-hoc inter-university network \cite{drioli2013networked}. 

%  For pedagogical applications, however, may involve other factors, such as the quality of sound, the sound environment, the spatial acoustics information. 
With the goal of using NMP in the educational context, however, the main focus is not on the objective or subjective quality of performance, but rather on those aspects and factors that can guide students to improve their technique or enable comfortable remote rehearsal. Moreover, in order to widen the audience of NMP, it needs to be available also for general purpose connections and hardware, hence addressing higher processing and transmission latency.  

On the one side, we need to take the influence of latencies into consideration to correctly design our tools for NMP. This involves to conduct perceptual tests to understand how these factors affect the sense of presence, and the quality of the performance. On the other side, we also aim at starting an open dialog with music teachers and students to collect comments and translate them into technical requirements for the project. 

In this paper, we present the architecture of the framework we will use for the perceptual tests and we describe the tests we have conducted on the temporal factors. We discuss the outcome of these tests in terms of preliminary perceptual and objective results and highlighting those comments from students that will guide our future work with the project.

The paper is organized as follows: in Section~\ref{sec:relatedwork} we provide the background context on top of which we are building our study; in Section~\ref{sec:framework} we present the designed architecture for perceptual experiments; Section~\ref{sec:experiment} introduces the pilot experiments; we discuss the preliminary observations from the experiments and next steps of the project in Section~\ref{sec:discussion}.

%In this paper, we introduce a study, still in progress, aimed at understanding how temporal factors (i.e., network latency) affect the sense of presence, and the quality of the performance of chamber music duos involved in remote collaboration, i.e music making. 
%
%We ask duos to perform a short exercise, under diverse conditions of network delay. 
%The exercise is specifically conceived around musical structures which are functional to pinpointing a set of problems relative to time management, communication mechanisms and mutual understanding between remote performers. A qualitative assessment through questionnaires on the sense of presence and the perceived quality of the performance~\cite{schubert2001experience} is combined with quality metrics of the objective performance~\cite{rottondi2015feature}.
%A follow-up study will be devoted instead to the investigation of spatial representations, auditory and visual.
%
%The premise is that effective music-making and communication rely on the availability of auditory and visual cues (i.e., sonic gestures)~\cite{godoy2010musical}, which are inevitably constrained in NMP, and in telepresence environments in general. 
%We turn the traditional engineering approach to NMP research upside down, and seek for design strategies to compensate, and facilitate a plausible music experience in the mediated environment. 
%
%In this respect, we are investigating how simulated network delay, and diverse modes of audio-visual spatial representation, separately, affect the subjective experience of being present and together in the shared reality environment. 
%Sensory breadth and depth, degree of control and anticipation of events, together with the overall interactivity of the environment, represent crucial elements in both presence and performance, being the first a prerequisite for the second~\cite{nash2000review}.
%

