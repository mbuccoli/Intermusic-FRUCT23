
\section{Introduction}\label{sec:introduction}

The rapid evolution of technology and the consequent increasing speed of digital communication networks allows to improve the communication experiences with the dramatically reduction of the virtual distances. Among the others, the process, facilitates the emerging of new scenarios of interaction among geogra\-phically-displaced people. Since the late '90s, a community composed of musicians, technicians and  scientists has been investigating how technology can enable music performances leading to the definition of Networked Music Performance (NMP), which \textit{occurs when a group of musicians, located at different physical locations, interact over a network to perform as they would if located in the same room} \cite{Lazzaro2001}.

NMP has immediate application in the performative scenario, enabling musicians to rehearse from remote distances and to perform in geographically-distributed concerts \cite{barbosa2003displaced}. As part of the broader category of audio-video streaming and conferencing systems, NMP has been interesting applications of educational purpose such as blended and distance learning \cite{IorwerthNMP2015}.    


% InterMUSIC: what it is, what it aims, how it is involved with NMP.
The EU funded project InterMUSIC\footnote{http://intermusicproject.eu/} (Interactive Environment for Music Learning and Practicing, 2017 - 2020) aims to develop and improve tools for distance learning of music and to collect such tools in integrated remote environments for music interaction and education. Within the project we have chosen three online pilot courses: music theory and composition, chamber music practice, and vocal training. Tools for courses are developed following two main paradigm. %Tools for course are developed by means of 

In the first paradigm, students from any part of the world access the course and find the teaching material provided by the teacher, with the possibility to interact with their colleagues and professors. This paradigm, namely Massive Open Online Courses, is designed for one-to-many asynchronous communication and it is widely used also by elite universities \cite{MOOCS}.

In the second paradigm, students use NMP softwares for attending master-student lessons or rehearse together. This paradigm has stricter requirements in terms of synchronicity and therefore only allow one-to-one (or few-to-few) synchronous communication. In this paper, we focus on the NMP-based paradigm and we discuss the  requirements related to the specific pedagogical scenario.

%Music Networked Performance is an interesting outcome of technology, and has receive attention under different factors.
The literature about NMPs has investigated numerous factors which may affect several aspects of the performances. Affecting factors include the unavoidable network latencies \cite{Chafe1}, timbral properties of the employed instruments \cite{Kolazi2013} and rhythmic properties of the performed pieces \cite{RottondiFeature}; while affected aspects range from objective quality of the performance \cite{Chafe3}, to perceptual metrics of musicians' comfortability \cite{CIM2018}. This investigation has led a highly popular tool for NMP, which allows low-latency interaction by using high-end hardware connected through ad-hoc inter-university network \cite{drioli2013networked}. 

%  For pedagogical applications, however, may involve other factors, such as the quality of sound, the sound environment, the spatial acoustics information.
In order to widen the audience of NMP, it needs to be available also for general purpose connections and hardware, hence addressing higher processing and transmission latency. Nevertheless, with the idea of using NMPs in the educational context, the goal of is not the NMPs' objective or subjective quality, but rather on those aspects and factors that can guide students to improve their technique or enable comfortable remote rehearsal. 

% we do not know which aspect to investigate--> we need experiments
The aspects we need to address for the success of pedagogical NMPs are not known a priori, leading us to make some assumptions. For example, we may assume that the acoustics of the environments affect or alienate the sense of presence of performers, or we may require that the networked environment should not worsen students' level of stress. In order to verify our assumptions, we need to conduct perceptual experiments with musicians to test different conditions and evaluate the result of these experiments. Back to the previous examples, we need to test how musicians perform in environments with different acoustics, or we need to measure musicians' stress level using sensors for biometric signals \cite{Yoshie2009}. While conducting these experiments, we also want to start an open dialog with music teachers and students to collect comments during the performances and translate them into technical requirements for the project. 

With this goal in mind, in this paper we identify the main entities involved in the performances and the interaction among them and we formalize them in a framework. The role of the framework is to provide an abstraction for the experiments that we need to conduct. The framework is designed to be generic for any kind of networked or physical performance, and we plan to extend it in order to provide a comprehensive semantic description of rehearsal and teaching activities for pedagogical scenarios. We present the architecture and details of the framework in Section \ref{sec:framework}.

Using this framework, we are able to design the perceptual experiments for conducting investigations on NMPs in the pedagogical scenarios. In Section ~\ref{sec:setup} we describe two pilot experiments we conducted to investigate the role of visual cues in musical interaction, and the influence of latency in the sense of presence of musicians, respectively. The preliminary results of the experiments and the next steps of our investigation are presented in  Section \ref{sec:discussion}.  

%on the temporal factors. We discuss the outcome of these tests in terms of preliminary perceptual and objective results and highlighting those comments from students that will guide our future work with the project.


%Moreover, we must continue to take the influence of latency into consideration to correctly design our tools for NMP.

%On the one side, we need to take the influence of latencies into consideration  This involves to conduct perceptual tests to understand how these temporal factors affect the sense of presence, and the quality of the performance. On the other side, we also aim at 


%In this paper, we present the architecture of the framework we will use for the perceptual tests and we describe the tests we have conducted on the temporal factors. We discuss the outcome of these tests in terms of preliminary perceptual and objective results and highlighting those comments from students that will guide our future work with the project.

%The paper is organized as follows: in Section~\ref{sec:background} we provide the background context on top of which we are building our study; in Section~\ref{sec:framework} we present the designed architecture for perceptual experiments;  steps of the project in Section~\ref{sec:discussion}.

%In this paper, we introduce a study, still in progress, aimed at understanding how temporal factors (i.e., network latency) affect the sense of presence, and the quality of the performance of chamber music duos involved in remote collaboration, i.e music making. 
%
%We ask duos to perform a short exercise, under diverse conditions of network delay. 
%The exercise is specifically conceived around musical structures which are functional to pinpointing a set of problems relative to time management, communication mechanisms and mutual understanding between remote performers. A qualitative assessment through questionnaires on the sense of presence and the perceived quality of the performance~\cite{schubert2001experience} is combined with quality metrics of the objective performance~\cite{rottondi2015feature}.
%A follow-up study will be devoted instead to the investigation of spatial representations, auditory and visual.
%
%The premise is that effective music-making and communication rely on the availability of auditory and visual cues (i.e., sonic gestures)~\cite{godoy2010musical}, which are inevitably constrained in NMP, and in telepresence environments in general. 
%We turn the traditional engineering approach to NMP research upside down, and seek for design strategies to compensate, and facilitate a plausible music experience in the mediated environment. 
%
%In this respect, we are investigating how simulated network delay, and diverse modes of audio-visual spatial representation, separately, affect the subjective experience of being present and together in the shared reality environment. 
%Sensory breadth and depth, degree of control and anticipation of events, together with the overall interactivity of the environment, represent crucial elements in both presence and performance, being the first a prerequisite for the second~\cite{nash2000review}.
%

