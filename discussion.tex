\section{Results and discussion}\label{sec:discussion}

\subsection{Evaluation}
We could observe different strategies of musical coordination and interpretation, based on breathing signaling and communicative gestures to keep synchronization, especially for attacks and the duration of sustained notes. In this case, the no sight condition deeply affected the expressiveness of the performance. In full visual occlusion, the performers rely mostly on acoustic cues to keep the tempo, with the apparent effect of a rather “gallopping” playing.

Concerning the musical sequences, they reported the more completeness and effectiveness of test 1, while they found the looping sequences quite easy, except for the change of 


The partial occlusion condition somewhat mimics a possible strategy for visual information reduction in remote performance (blob). 


\subsection{Future directions}


