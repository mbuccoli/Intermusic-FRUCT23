
\section{Conclusion}\label{sec:conclusion}
%This part presents the key findings in substance. Avoid simple enumeration of the following material. It is desirable to provide a link to existing articles and grants. 

Using Networked Music Performances for pedagogical scenarios requires a deep investigation on the topic in order to find metrics, factors and aspects that may help musicians to improve their musical skills.

In this paper, we introduced a framework for conducting perceptual experiments to continue this investigation for the purposes of the project InterMUSIC. We then presented two experiments conducted using the framework and the preliminary results that we observed.

Starting from these results, we described the areas of investigation that we intend to follow, with the final goal of developing a platform for NMPs in pedagogical scenarios that also work with general-purpose hardware.

Beyond this area, in future work we intend to further develop the formalization of the framework into an ontology, which will be integrated in the NMP tools, in order to collect and analyze a number of semantically-annotated rehearsal or lessons \cite{Kolazi2013}. 