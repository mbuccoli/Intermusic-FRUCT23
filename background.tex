\section{Background and Related Work}\label{sec:background}
A wide investigation on NMPs has been devoted to the influence of different latency conditions on the performance. We discuss the main results in Section \ref{subsec:temporal}. In Section \ref{subsec:network} we briefly introduce the investigation regarding the network topologies and architecture, while in Section \ref{subsec:acoustics} we provide an overview of the studies on rooms' acoustic factors. Lastly, in Section \ref{subsec:sw} we list the main softwares employed for NMPs and their main features.

\subsection{Temporal factors}\label{subsec:temporal}
Temporal factors refer to every aspect of the infrastructure that cause the presence of end-to-end delay between the musicians present in different locations. We can broadly divide the causes of end-to-end delay into two main factors, i.e., the signal processing delay and the network delay \cite{Lakiotakis}. 

The former comprises the delays caused by the whole signal chain, consisting of acquisition hardware (e.g. soundcards), the encoding/decoding and fragmentation processes. The latter comprises all the possible delays caused by the network transmission between transmitter and receiver due for example to network congestion.

The level of latency is a factor that dramatically affects the NMP experience and as such was extensively analyzed in the literature. In~\cite{RottondiFeature} the authors conduct a series of content-based experiments that show how high latency values in the range of $20-60~\mathrm{ms}$ cause a decrease in the quality of the performance expressed by a tendency of the musicians to tempo deceleration. 

In~\cite{Chafe1,Chafe2,Chafe3} the authors take into account the concept of \textit{Temporal Separation}, which represents the time needed for the action of one person to reach another one in a setting where both are acting together. A set of experiments are devised, which evaluate the ability of several couples subjects in performing a clapping rhythm together while being in two separated rooms and while varying the transmission latency in the range of 3-78 ms. The results of such experiments prove that the best performance results are obtained when the transmission latency between the subjects is comparable to a Temporal Separation value corresponding to a setting where the two subjects are in the same room. The authors also observed that unnaturally low latency values cause an acceleration in the musicians' tempo.

%When confronted with latency values different from a \textit{Temporal Separation} corresponding to a physical setting, the musicians devise a set of techniques in order to cope with such difficulty. Such techniques are identified in~\cite{Carot07networkmusic} and vary depending on the latency level. If the latency is less than $25\mathrm{ms}$ the musicians are able to follow a Realistic Interaction Approach (RIA), which corresponds to the delay that would have been sensed if they were in the same room. This is the only type of approach considered acceptable for professional musicians. When latency exceeds $25\mathrm{ms}$ the subjects find different ways to cope with latency, depending on the scenario and on the type of music. In the Master Slave (MSA) and Laid Back (LBA) approaches one of the two musicians follows its own tempo, while the second one follows. MSA is appropriated when two rhythm instruments are performing, since one can simply follow the rhythm of the other, instead, in the case of LBA we usually have a rhythmic and solo instrument playing, the rhythmic one mantains the tempo, while the solo one performs slightly out of time. The Delayed Feedback approach tries to make the musicians as if they were in a RIA setting, by delaying artificially the signal of one of the players.

\subsection{Network factors}\label{subsec:network}
The choices regarding all the aspects of the network architecture depend on the aspects needed in the NMP framework. 
Network architecture used for NMP comprise both decentralized peer-to-peer(P2P) and client-server~\cite{RottondiOverview}. 

In P2P architectures each participant has to send its audio/video data to all the other musicians. This poses great issue for what concerns the scalability of the NMP system, causing a trade-off between the number of people connected and the quality of the audio-video content, which must be degraded in order to gain enough bandwidth to connect all the musicians.

Client-server architectures address the scalability issue, since they are based on a server that receives the individual streams sent by each musicians, mixes it and then sends it back. However, the bandwidth requirements remain rather high, and the server causes an additional delay, since it adds a two-way communication with each participant.


\subsection{Acoustic and Spatial factors}\label{subsec:acoustics}
The analysis of the impact of acoustic and spatial factors in NMP and distance learning has not been as deep and thorough as the ones performed on the analysis of the tempo effects. 

%The effect of the acoustic environment were considered in few works. 
In~\cite{carot2009towards} the authors performs a set of experiments in semi-anechoic chambers and add different levels of artificial reverb in order to simulate a natural environment. The experiments show that the performances of the musicians provide no noticeable differences that could be directly explained by the change in reverberation. 

In \cite{FarnerReverb}, the author present an experiment with two musicians performing handclapping (similarly to \cite{Chafe1}) considering three acoustic conditions: real reverberant, virtual anechoic and virtual reverberant. The results show that anechoic conditions cause higher imprecisions than the reverberant ones.

In~\cite{gurevich2011ambisonic} the authors present a study that concerns the application of Ambisonics techniques to NMP problems. Athough they do not develop a fully functional framework for NMP, they demonstrate the feasibility of implementing 3D spatial audio for NMP tasks. %which may be one of the field most future developments.


\subsection{Available softwares}\label{subsec:sw}
In this Section we list the main tools that have been developed for NMP; for an extensive list we refer the reader to ~\cite{RottondiOverview}.

JackTrip~\cite{CaceresJacktrip} was developed by the SoundWIRE research group at CCRMA in order to support bi-directional music performances. It is based on uncompressed audio transmission through high-speed links such as \textit{Internet2}. In the current version, it does not support video transmission.

The LOLA\cite{drioli2013networked} project was developed by the Conservatory of Music G. Tartini in Trieste in collaboration with the Italian national computer network for universities and research (GARR). LOLA is based on low-latency audio/video acquisition hardware and on the optimization of all the steps needed to transmit audio/video contents through a dedicated network connection. Because of  its low-latency properties, we used it in the preliminary InterMUSIC experiments presented in \cite{CIM2018}. As a drawback, the project is not open source and it is not optimized for generic network connections.

On the other side, UltraGrid~\cite{holub2006high} is an open-source software that allows audio/video low latency transmission. While its performance are still far from those achieved by LOLA, it is more flexible for generic hardware and networks and it allows contributors to implement  new functionalities. For this reason, we are considering the use of UltraGrid in the next stages of the InterMUSIC project.