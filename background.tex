\section{Background and Related Work}\label{sec:background}

introduction on the different factors seen, then some literature

\subsection{Temporal factors}
Temporal factors are of major importance in determining the quality of the NMP experience. They refer to every aspect of the infrastructure that cause the presence of end-to-end delay between the musicians present in different locations. 
We can broadly divide the causes of end-to-end delay into two main factors, which consist of the audio processing delay and the network delay \cite{Lakiotakis}. The first factor comprises the delays caused by the whole audio chain, consisting of the audio hardware (e.g. soundcards), and the encoding/decoding and fragmentation processes. The second factors comprises all the possible delays caused by the network transmission between transmitter and receiver due for example to network congestion.
The level of latency is a factor that dramatically changes the NMP experience and as such was extensively analysized in the literature. In~\cite{RottondiFeature} the authors conduct a series of signal-based experiments that show how high latency values in the range of $20-60~\mathrm{ms}$ cause a decrease in the quality of the performance expressed by a tendency of the musicians to tempo deleceration. 

In~\cite{Chafe1,Chafe2,Chafe3} the authors take into account the concept of \textit{Temporal Separation}, which represents the time needed for the action of one person to reach another one in a setting where both are acting together. In the case of two musicians playing together, the temporal separation can be defined in terms of space and medium between the musicians. A set of experiments are devised, which evaluate the ability of several couples subjects in performing a clapping rhythm together via headphones, while being in two separated rooms and while varying the transmission latency in the range of $3-78~\mathrm{ms}$. The results of such experiments prove that the best performance results are obtained when the transmission latency between the subjects is comparable to a \textit{Temporal Separation} value corresponding to a setting where the two subjects are in the same room. Unnaturally low latency values cause a acceleration in the musicians' tempo, on the contrary, unnaturally high values cause the musicians to accelerate.
When confronted with latency values different from a \textit{Temporal Separation} corresponding to a physical setting, the musicians devise a set of techniques in order to cope with such difficulty. Such techniques are identified in~\cite{Carot07networkmusic} and vary depending on the latency level. If the latency is less than $25\mathrm{ms}$ the musicians are able to follow a Realistic Interaction Approach (RIA), which corresponds to the delay that would have been sensed if they were in the same room. This is the only type of approach considered acceptable for professional musicians. When latency exceeds $25\mathrm{ms}$ the subjects find different ways to cope with latency, depending on the scenario and on the type of music. In the Master Slave (MSA) and Laid Back (LBA) approaches one of the two musicians follows its own tempo, while the second one follows. MSA is appropriated when two rhythm instruments are performing, since one can simply follow the rhythm of the other, instead, in the case of LBA we usually have a rhythmic and solo instrument playing, the rhythmic one mantains the tempo, while the solo one performs slightly out of time. The Delayed Feedback approach tries to make the musicians as if they were in a RIA setting, by delaying artificially the signal of one of the players.

\subsection{Network factors}
The choices regarding all the aspects of the network architecture depend on the aspects needed in the NMP framework. 
Network architecture used for NMP comprise both decentralized peer-to-peer(P2P) and client-server~\cite{RottondiOverview}. In P2P architectures each participant has to send its audio/video data to all the other musicians. This poses great issue for what concerns the scalability of the NMP system, causing a trade-off between the number of people connected and the quality of the audio-video content, which must be degraded in order to gain enough bandwith to connect all the musicians.
Client-server architectures solve the scalability issue, since they are based on a server that receives the individual streams sent by each musicians, mixes it and then sends it back. However, the server still needs a great amount of bandwith in order to be able to handle the received streams. Also the addition of the server causes an additional delay, by adding a path back and forth from each partecipant to the server.


\subsection{Latency factors}
\subsection{Acoustic and Spatial factors}
The analysis of the impact of acoustic and spatial factors in NMP and distance learning has not been as deep and thorough as the ones performed on the analysis of the tempo effects. 

The effect of the acoustic environment were considered in a few works. In~\cite{carot2009towards} the authors performs a series of test in semi-anechoic chambers and add different levels of artificial reverb in order to simulate a natural environment, however their experiments show that the performances of the musicians provide no noticeable differences that could be directly explained by the change in reverberation. In \cite{FarnerReverb} perform handclapping experiments, similar to the ones considered in~\cite{Chafe1} considering two rooms and real reverberant, virtual anechoic and virtual reverberant conditions. Their results show that anechoic conditions cause higher imprecisions than the reverberant ones.



In~\cite{gurevich2011ambisonic} the authors present a study that concerns the application of ambisonics techniques to NMP problems. Athough they do not develop a fully functional framework for NMP, they demonstrate the feasibility of implementing 3D spatial audio for NMP problems, which may be one of the field most future developments.


\subsection{Available softwares}
Several softwares able to support NMPs have been developed in the course of the years and a extensive list is reported in~\cite{RottondiOverview}.
JackTrip~\cite{CaceresJacktrip} was developed by the SoundWIRE research group at CCRMA in order to support bi-directional music performances. It is based on uncompressed audio transmission through high-speed links such as \textit{Internet2}. Since does not support video transmission, it is not suited for our purposes.
The LOLA\cite{drioli2013networked} project was developed by the Conservatorio di Musica Giuseppe Tartini of Trieste in collaboration with GARR (Gruppo per l'Armonizzazione delle Reti della Ricerca). LOLA is based on low-latency audio/video acquisition hardware and on the optimization of all the steps needed to transmit audio/video contents. It allows to achieve a low round trip delay and as such it has been utilized in several music related performances, however it is not open source and needs to be connected to an academic network in order to work properly. Due to its low-latency properties it was used in the preliminary InterMUSIC experiments presented in \cite{CIM2018}. 
UltraGrid~\cite{holub2006high} is a software that allows audio/video low latency transmission, while it is still not able to reach the same results as LOLA, its open source nature, makes easier the implementation of new functionalities and for such reasons it will be considered in future experiments in the scope of InterMUSIC.