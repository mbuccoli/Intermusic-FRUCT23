\begin{abstract}
Networked Music Performance is a challenging task that has been mainly investigated for their performative application.	For the InterMUSIC project, we aim to investigate it from a pedagogical perspective and use it as a helpful tool for distance learning of music. In this paper, we introduce a framework for conducting experiments for Networked Music Performance and we present two preliminary experiments on the sense of presence of musicians during a networked performance. We discuss the comments collected by the musicians as a set of requirements and guidelines for future experiments.
\end{abstract}
