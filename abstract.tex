\begin{abstract}
With the big improvement of digital communication networks, Networked Music Performances (NMPs) received a great interests from music live performance e music recording industry. The positive impact of NMPs in pedagogical applications, instead, it is not still not explored. Within the InterMUSIC project, we aim to investigate NMPs from a pedagogical perspective, that has important differences with respect to music performances, and to develop tools to improve distance learning experiences. In this paper, we introduce a conceptual framework for conducting experiments for Networked Music Performance and we present two preliminary experiments on the sense of presence of musicians during a networked performance. We discuss the comments collected by the musicians as a set of requirements and guidelines for future experiments.
\end{abstract}
